
The first step of the design process is to identify the requirements for the wafer projecting system we are considering. In order to define clear requirements, we have first identified the constituent components of the system and two main operating sequences.

This chapter describes the main components of the system, the manner in which these components form processes inside the system and the global requirements of the system.

\section{System Components}

\begin{itemize}
\item An UV lamp L
\item A vacuum chamber
\item Two airlocks A1, A2

\item Two sets of doors
    \begin{itemize}
    \item Input doors: DI1, DI2
    \item Output doors: DO1, DO2
    \end{itemize}

\item Three robots
    \begin{itemize}
    \item Outside the vacuum chamber: R1, R2
    \item Inside the vacuum chamber: R3
    \end{itemize}

\item Two sets of wafer stacks
    \begin{itemize}
    \item Input stacks: I1, I2
    \item Output stacks: O1, O2
    \end{itemize}
    
\item One lamp sensor - Lamp Empty, Lamp Full
\item Two airlock sensors - Airlock Empty, Airlock Full
\item Four stack sensors - Stack Empty, Stack Full
 
\end{itemize}

\section{Sequences}

We have identified two main sequences of the process. The first sequence describes the steps needed to move a blank wafer from the input stack to the lamp, while the second sequence describes how a wafer reaches the output stack after being projected by the lamp.

\subsection{Wafer to lamp}
\begin{enumerate}
    \item Outside robot picks up wafer from input stack
    \item Airlock output door opens
    \item Robot drops wafer inside airlock
    \item Airlock output door closes
    \item Airlock input door opens
    \item Inside robot picks up wafer from airlock
    \item Inside robot drops wafer to lamp
    \item Lamp turns on and projects the wafer
\end{enumerate}

\subsection{Wafer to output stack}
\begin{enumerate}
    \item Lamp turns off
    \item Inside robot picks up wafer from lamp
    \item Inside robot drops wafer inside airlock
    \item Airlock input door closes
    \item Airlock output door opens
    \item Outside robot picks up wafer from airlock
    \item Outside robot drops wafer to output stack
\end{enumerate}

\section{Safety Requirements}

Based on the sequences described above, we have identified a set of requirements, which were grouped with respect to the components we previously distinguished. 

\begin{enumerate}
\item Airlocks:
    \begin{enumerate}
    \item DI1/DI2 can only be opened if DO1/DO2 is closed.
    \item DO1/DO2 can only be opened if DI1/DI2 is closed.
    \end{enumerate}

\newpage

\item Output stacks:
    \begin{enumerate}
    \item A wafer can only be placed into the output stack only if it was projected by the lamp.
    \end{enumerate}

\item Lamp:
    \begin{enumerate}
    \item The lamp can project only a wafer at a time.
    \item The lamp can project a wafer only once.
    \end{enumerate}
    
\item Robots:
    \begin{enumerate}
    \item Robots can pick-up/drop only one wafer at a time.
    \item R1/R2 can only drop a wafer into O1/O2 if the stack is not full.
    \item R1/R2 can only pick-up a wafer from I1/I2 if the stack is not empty.
    \item R1/R2 can only pick-up a wafer from A1/A2 if DO1/DO2 is open.
    \item R1/R2 can only drop a wafer in A1/A2 if DO1/DO2 is open.
    \item R3 can only pick-up a wafer from A1/A2 if DI1/DI2 is open.
    \item R3 can only drop a projected wafer in A1/A2 if DI1/DI2 is open.
    \item After picking-up a wafer from A1/A2, R3 can only drop it into the lamp.
    \item After picking-up a wafer from the lamp, R3 can only drop it into A1/A2.
    \item R1 can pick-up wafers from I1 and drop them only into A1.
    \item R2 can pick-up wafers from I2 and drop them only into A2.
    \item R1 can pick-up wafers from A1 and drop them only into O1.
    \item R2 can pick-up wafers from A2 and drop them only into O2.
    \end{enumerate}
\end{enumerate}

\section{Liveness Requirements}

The following functional requirements ensure the movement of the wafers throughout the system.

\begin{enumerate}
    \item From the input stack, a wafer should be able to reach the lamp.
    \item From the lamp, a wafer should be able to reach the output stack.
\end{enumerate}