\documentclass[]{report}


% Title Page
\title{}
\author{Aniket Ashwin Samant}


\begin{document}
\section*{Reflection Report}
The System Validation project was a good example to demonstrate the fact that formally modelling and verifying even a simple robotic system with minimal implementation details can be a challenging task, and hence why it is highly necessary to make use of formal modelling and verification methods when designing large, complex systems. This project certainly opened my mind to the notion of what it truly means to verify a system without diving into its implementation details so as to ensure that there are no fundamental design errors at the conceptual modelling stage itself.\\

As my group would be discussing ideas for our project, it would often happen that we'd all agree on a certain idea (say, for instance, that the airlock doors simply \textit{can't} be open at the same time based on the other conditions set by us) but after simulating the conditions and steps using mcrl2 we'd find out that the semantic domain generally tends to hide quite a few logical details. In order to have the system working as desired, we'd have to think in terms of the components working in parallel, which is an uphill task unless we start from simple models and tackle actions one by one. Even then, there'd be corner cases lurking around for which we'd have to think deeper to resolve them.\\

Getting things right was quite a challenge, but since we worked together as a group, we were able to overcome them (albeit not very smoothly, I must confess, given time constraints). Since none of us stays close to the university either, we had to schedule lots of meetings in various locations across the campus to get together and discuss the project. To be really fair, it wouldn't do justice to say that any one person worked more than the others in the group (while I do agree that objectively the numbers would not come to exactly 25$\%$, I'd say they're between 22 to 28 $\%$). At least from a personal view, I believe we all put in equal effort.\\

The most important aspect I gathered from the project, I feel, is to acknowledge that everyone thinks in a different way and the best way to get things moving is to have discussions on contentious topics and weigh the tradeoffs objectively to make sure everyone's on the same page. The other approaches may be noted down, but not dismissed entirely so that they can be revisited later if the selected approach does not work.\\

Having worked in a professional environment involving large teams before coming here, I was able to apply some principles of teamwork (and project management too) learnt there to my group project, and it did feel very rewarding. To sum it up, working on the project was a good learning experience, not only from its technical perspective, but also on the "teamwork"-developing front. Moreover, this being the first project for me in my Master's program here, it has set a good precedent for further group projects and assignments to come. 

 
\end{document}          
